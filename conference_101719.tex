\documentclass[conference]{IEEEtran}
\IEEEoverridecommandlockouts



% 日英切り替え
\newif\ifjp
\jptrue
% \jpfalse
\ifjp
\usepackage{luatexja-preset}
\fi

\usepackage[dvipdfmx]{graphicx}
% \usepackage{cite}  % 参考文献の参照を "[2,3,1]" から "[1--3]" または "[2], [3], [1]" から "[1]--[3]" に修正,英文での参照 "[1]" の前に空白がない場合に空白を挿入(和文では標準で挿入される)
\usepackage{url}
\usepackage{tikz}
\usepackage{booktabs}% 表の横罫線の改善版コマンド(\toprule, \midrule, \cmidrule, \bottomrule)
\usepackage{hhline}% 表の二重罫線を拡張
\usepackage{multirow}% 表の縦二段をつなげる \multirow コマンド
\usepackage{tabularx}% 横幅を指定可能な表の tabularx 環境
\usepackage{eqparbox}% 自動的に同じ幅に調整されるボックス(表など)の \eq par box コマンド
\usepackage{fixltx2e}
\usepackage{array}
\usepackage[cmex10]{amsmath}% AMS の数式環境(amstext.sty, amsopn.sty, amsbsy.sty を含む),cmex10 オプションでビットマップフォントの埋め込みを防ぐ
\usepackage{amssymb}% AMSFonts(数学記号)を使用
% \usepackage{amsthm}  % 定理型の環境を定義する \newtheorem コマンドを拡張,証明を書く proof 環境
\usepackage{caption}
% \usepackage{subcaption}
\usepackage{widetable}
\usepackage{makecell}
% \usepackage{threeparttable}
\usepackage[normalem]{ulem}
\useunder{\uline}{\ul}{}
\usepackage{mathtools}
\usepackage{bm}% 数式用ボールドフォントの \bm コマンド(amsbsy.sty の改善版),txfonts.sty や mathptmx.sty の前で読み込むと boldmath が Computer Modern になってしまうので注意
% \usepackage{flushend}

\newcommand{\lrabs}[1]{\left\lvert#1\right\rvert}
\newcommand{\lrnorm}[1]{\left\lVert#1\right\rVert}

% 図表を記載した位置の近くに出力する(後のページに追いやられるのを防ぐ)ためのフロートのパラメータ設定
\renewcommand{\topfraction}{1.0}% ページ上部のフロートが占める上限 [t]
\renewcommand{\dbltopfraction}{1.0}% ページ上部のフロートが占める上限 [t](2カラムの*版)
\renewcommand{\bottomfraction}{1.0}% ページ下部のフロートが占める上限 [b]
\renewcommand{\textfraction}{0.0}% ページの本文が占める割合の下限
\renewcommand{\floatpagefraction}{0.8}% 出力位置 [p] が実行されるのに必要な最低限のフロート占有比率
\renewcommand{\dblfloatpagefraction}{0.8}% 出力位置 [p] が実行されるのに必要な最低限のフロート占有比率(2カラムの*版)
\setcounter{topnumber}{10}% ページ上部のフロートの上限数 [t]
\setcounter{dbltopnumber}{10}% ページ上部のフロートの上限数 [t](2カラムの*版)
\setcounter{bottomnumber}{10}% ページ下部のフロートの上限数 [b]
\setcounter{totalnumber}{20}% ページあたりのフロートの上限数

\usepackage{circledsteps} % 番号付きの丸囲み文字を使うためのパッケージ

\usepackage{cite}
\usepackage{amsmath,amssymb,amsfonts}
\usepackage{algorithmic}
\usepackage{graphicx}
\usepackage{textcomp}
\usepackage{xcolor}
\def\BibTeX{{\rm B\kern-.05em{\sc i\kern-.025em b}\kern-.08em
T\kern-.1667em\lower.7ex\hbox{E}\kern-.125emX}}
\begin{document}

\captionsetup{skip=2pt} % 図表キャプションと本文の間隔を調整

\title{\LARGE \bf Topological Mapping with Visual Place Recognition \\
using Panorama Shift of 360-degree Images \\
and Spatial Consistency Check\\
% {\footnotesize \textsuperscript{*}Note: Sub-titles are not captured in Xplore and
% should not be used}
% \thanks{Identify applicable funding agency here. If none, delete this.}
}

\author{\IEEEauthorblockN{Hiroki Tsukoshi}
\IEEEauthorblockA{\textit{Graduate School of Science and} \\
\textit{Technology},
\textit{Meiji University},\\
Kanagawa, Japan}
\and
\IEEEauthorblockN{Yoshitaka Hara}
\IEEEauthorblockA{\textit{Future Robotics Technology Center(fuRo)},\\
\textit{Chiba Institute of Technology},\\
Chiba, Japan}
\and
\IEEEauthorblockN{Yoji Kuroda}
\IEEEauthorblockA{\textit{School of Science and}\\
\textit{Technology},
\textit{Meiji University}\\
Kanagawa, Japan}
}

\maketitle

\begin{abstract}
This paper proposed a method for topological mapping with visual place recognition using panoramic image shift and spatial consistency check.
In our previous work, We have been working on constructing topological maps that align with the phase structure of the real environment. However, due to the nature of conventional algorithms, loop detection may fail when revisited locations intersect or are in reverse direction along the robot's travel path during data acquisition.
This method involves shifting panoramic image sequences acquired from a spherical camera according to a specific criterion, then extracting features from the shifted images.
Next, nodes and arcs are created based on the visual similarity of the extracted feature vectors.
By detecting loops while considering spatial consistency for these nodes and arcs, a topological map is constructed that aligns the real environment with the topological structure.
Based on experimental results in both indoor and outdoor environments, the proposed method constructs a topological map with correct phase structure.
This is achieved by improving visual place recognition accuracy independent of the driving path through panoramic image shift, and by suppressing false loop detections via spatial consistency check.
\end{abstract}

\begin{IEEEkeywords}
Topological Map, Visual Place Recognition, Panoramic Image
\end{IEEEkeywords}

% \section{Introduction}
\section{緒言}
移動ロボットの自律走行に用いる地図には,メトリックマップとトポロジカルマップがある.メトリックマップとは点群地図や占有格子地図のことを指し,環境を座標系上で表現する.
一方でトポロジカルマップは,環境の接続構造をノードとアークで抽象的に表現する.

これまで筆者らは,実環境との位相構造の整合性を持ち,また接続方向の正確性も備えたトポロジカルマップの構築に取り組んできた\cite{ref:robosym_ogiwara}\cite{ref:SII2026_nakao}.
前稿\cite{ref:robosym_ogiwara}では,全天球カメラで取得した全周画像から,Visual Place Recognition (VPR) 手法である AnyLoc\cite{ref:AnyLoc}で特徴抽出を行い,画像間の視覚的類似度をもとに適応的なノード作成とループ検出を行った.
しかし前稿\cite{ref:robosym_ogiwara}ではそのアルゴリズムの特性上,データ取得時のロボットの走行経路において,再訪地点が交差する場合や逆方向の場合では,ループ検出ができない場合があった.

本稿では,前稿\cite{ref:robosym_ogiwara}からの変更点として,全天球カメラで取得した各画像を一定の基準でパノラマシフトする.
また,ループ検出アルゴリズムを変更し,空間的一貫性を考慮したループ検出を行う.
これにより,再訪地点が交差・逆方向の場合においても,パノラマシフトによる視覚場所認識の精度向上と,空間的一貫性チェックによるループ検出の誤検出抑制を通じて,実環境との位相構造の整合性を持つトポロジカルマップを構築する.


本稿の貢献は,以下の通りである.
\begin{itemize}
  \item 全天球カメラで取得した全周画像に対して,方位についてのパノラマシフトを行うことで,進行方向が異なる再訪地点での視覚場所認識を可能とする.
  \item さらに空間的一貫性チェックを用いることで,ループ検出の誤検出を抑制し,実環境との位相構造の整合性を持つトポロジカルマップを構築する.
  \item 屋内環境と屋外環境での実験により,走行経路が交差する場合や逆方向の場合においても,正しい位相構造のトポロジカルマップを構築できることを示す.
\end{itemize}

% \section{Ease of Use}
\section{関連研究}

% \subsection{Maintaining the Integrity of the Specifications}


\section{視覚場所認識によるトポロジカルマップの構築}

\begin{figure*}[t]
  \centering
  \includegraphics[width=\linewidth]{fig/pipeline_and_concept.pdf} \vspace*{-8mm}
  \caption{提案手法のパイプラインと概念図}
  \label{fig:pipeline_and_concept}
\end{figure*}

\begin{figure*}[t]
  \centering
  \includegraphics[width=\linewidth]{fig/panorama_shift_concept.pdf} \vspace*{-8mm}
  \caption{パノラマシフトの概念図}
  \label{fig:panorama_shift_concept}
\end{figure*}

\subsection{概要}\label{AA}

図\ref{fig:pipeline_and_concept}に,提案手法のパイプラインと提案手法全体の概念図を示す.
入力は全天球カメラから取得した全周画像であり,再訪地点が交差・逆方向の場合においても,視覚場所認識を可能にするために用いる.
まず,事前に収集した動画データから,画像列を作成する.
次に,画像列に対して,パノラマシフトを行う.
その後,パノラマシフトされた画像列から特徴ベクトルを抽出し,それらの視覚的類似度を基に,ノードとアークを作成する.
最後に,空間的一貫性チェックを用いたループ検出を行い,トポロジカルマップを構築する.

\subsection{全周画像のパノラマシフトを用いた特徴抽出}

入力された動画形式の全周画像を一定の時間間隔でサンプリングし,画像列を作成する.
次に,すべての画像に対して,一定の基準でパノラマシフトを行う.

図\ref{fig:panorama_shift_concept}に,パノラマシフトの概念図を示す.
画像内の各縦列の特徴量を比較し,最も高い列を画像の左端に来るようにシフトする.
この処理により,同一地点に異なる方向から再訪した場合でも,画像を構成する物体の位置関係が変わらない画像を取得することができ,視覚的類似度の向上が期待できる.

次に,パノラマシフトされた画像列から特徴ベクトルを抽出する.
提案手法では,特徴抽出器として AnyLoc-VLAD-DINOv2\cite{ref:AnyLoc} を用いる.

\subsection{ノードとアークの作成}

% 提案手法では,前稿\cite{ref:robosym_ogiwara}と同様に視覚的類似度を基にノードとアークを作成する.
画像間の視覚的類似度を基に,ノードとアークを作成する.
% 類似度を用いて作成するノードを選定することで,画像間の視覚的変化が少ない場合にはノードが作成されず,ロボットが停止している場合や,視覚的に単調な環境を走行している場合にノードが過剰に作成されることを防ぐ.
視覚的類似度の評価には,各特徴ベクトル間のコサイン類似度を用いる.
類似度に対して閾値を設定し,とある画像に対しそれ以降の画像との類似度が閾値以下になると新たにノードを作成し,ノード間にアークを作成する.
次の回では,直前に作成したノードの特徴ベクトルを基準とし,同様の処理を繰り返すことでロボットの走行経路に沿ったノードとアークが作成される.
% 次の回では,基準となる特徴ベクトルを $\bm{v}_j$ とし,同様の処理を繰り返すことでロボットの走行経路に沿ったノードとアークが作成される.
式\eqref{eq:nodes}に,作成されたノード列を示す.
\begin{align}
  N = \{\bm{n}_1, \bm{n}_2, \ldots, \bm{n}_n\}
  \label{eq:nodes}
\end{align}
ここで,$n$は作成されたノードの総数を表す.

% 提案手法では,前稿\cite{ref:robosym_ogiwara}よりも $node\_sim\_th$ を高く設定し,ノード間の距離を短くする.
% ノード間隔を短くすることで,同一地点を表現するノードを増やすことができ,後述するループ検出でのクエリ画像と類似度の高い画像を増やすことを目的としている.

\subsection{空間的一貫性チェックを用いたループ検出}

% 単一画像のコサイン類似度だけでは,類似している画像が取得できる別地点との誤検出が発生しやすい.
% また前稿\cite{ref:robosym_ogiwara}では,そのアルゴリズムの特性上,同一地点を異なる方向から走行する場合はループ検出できない場合がある.
% 上記のようなループの誤検出や未検出により,実環境と位相構造が異なるトポロジカルマップが構築される.

提案手法では,空間的に連続したノードを用いたループ検出を行うことで,位相的に正しいトポロジカルマップの構築を行う.
ループ検出の基本的な考えとして,スライディングウィンドウを用いる.
% ウィンドウには事前に決められた数のノードが格納され,ノード列に対してウィンドウを一つずつずらしながら走査する.
ウィンドウには事前に決められた数のノードが格納され,それらとクエリノードとの類似度を計算する.
ノード列に対してウィンドウを一つずつずらしながら走査し,ループ検出対象のノードを特定する.
% 短い間隔で作成されたノードをウィンドウで走査することで,同一地点を異なる方向から走行した場合でも,ウィンドウ内にクエリと類似度の高いノードが多く含まれ,空間的一貫性を持つループ検出が可能になると考えられる.

% はじめに,クエリとなるノードとクエリ自身も含めた全ノード間との類似度を計算する.
% この処理をすべてのノードで行うことで、各ノードがクエリになった場合の類似度をあらかじめ算出したことになる.

% クエリノードを $\bm{n}_q$ として,全ノード列に対してスライディングウィンドウを用いて走査する.
% 以降,$\bm{n}_q$ と類似度を比較する対象のノードをターゲットノード $\bm{n}_t$ と表記する.
% クエリノード $\bm{n}_q$ に対して類似度を計算するノードをターゲットノード $\bm{n}_t$ とする.
クエリノードを $\bm{n}_q$ ,それに対し類似度を計算するノードをターゲットノード $\bm{n}_t$ とする.
まず,ウィンドウ内のノードのうち,$\bm{n}_q$ とループ検出に対する閾値以上の類似度をもつ $\bm{n}_t$ をカウントする.
次に,一つのウィンドウに対するカウントがカウント用の閾値を超えた場合,そのウィンドウをループ検出の候補ウィンドウとする.
最後に,候補ウィンドウ内で $\bm{n}_q$ と最も類似度の高い $\bm{n}_t$ をループ検出の候補ノード $\bm{n}_c$ とし,ノードペア $(\bm{n}_q, \bm{n}_c)$ を保存する.
この流れをすべてのノードをクエリノードとして行う.

ウィンドウ内には $\bm{n}_q$ 自身や,$\bm{n}_q$ と時間的に連続した近傍のノードが含まれる場合があり,それらがループ検出されることを防ぐ必要がある.
そのため,ウィンドウ内に $\bm{n}_q$ 自身と,$\bm{n}_q$ の近傍ノードが含まれる場合は,そのウィンドウでの処理をスキップする.
% スキップする範囲は,クエリノード $\bm{n}_q$ の前後 $neighbor\_node\_num$ 分のノードとする.
% 以下にここまでの処理で用いるパラメータを示す.
% \begin{itemize}
%   \item $loop\_sim\_th$ : クエリノードとターゲットノードのループ検出に対する類似度の閾値
%   \item $window\_size$ : 一つのウィンドウに含まれるノード数
%   \item $count\_th$ : ウィンドウ内で $loop\_sim\_th$ を超えたターゲットノードの数
%   \item $neighbor\_node\_num$ : ループ検出処理を省略するクエリノードからの近傍の範囲
% \end{itemize}

最後に,保存されたループ検出のノードペアに対して相互チェックを行い,ループ検出を確定する.
ここまでの処理から,${\bm{n}_c}$ もまたクエリノードになるため,$(\bm{n}_q, \bm{n}_c)$ と $(\bm{n}_c, \bm{n}_q)$ の両方が保存されている場合にループ検出を確定し,ノードペア間にアークを追加する.
% ループ検出の双方向からの検証を行うことで,誤検出を抑制しトポロジカルマップの位相構造が崩れることを防ぐ.

これらの処理により,ループ検出の誤検出を抑制しつつ位相的に正しいトポロジカルマップを構築できると考えられる.

\section{実験}
\subsection{実験条件}
\label{sec:exp_conditions}

図\ref{fig:env_images}に,各実験環境と走行経路を示す.
実験環境および走行経路は,屋内交差,屋内逆方向,屋外交差の3種類である.
図\ref{fig:env_images}(a) の赤色の経路は屋内交差,青色の経路は屋内逆方向である.
% 構築されたトポロジカルマップの定性評価により,提案手法の有効性を評価した.

WHILL Model CR をベースとした後輪差動駆動の車輪ロボットに RICOH THETA S を搭載し,各環境を走行させ全周画像を取得した.
すべてのデータ処理には,CPU: Intel Core i9-14900KF 5.700GHz,RAM: 64GB,GPU: NVIDIA GeForce RTX 4090,VRAM: 24GB を搭載した PC を用いた.

検証用データとして,メトリックな情報も取得した.
% メトリックな情報の取得には,3D-Lidar(Velodyne VLP-32MR),IMU(Xsense MTi-30-2A8G4)を用いた.
% 具体的には,車輪オドメトリの並進速度とIMUの回転角速度を統合した値と,3次元点群地図を事前地図とした NDT Scan Matching\cite{ref:ndt} の値を拡張カルマンフィルタでさらに統合した結果をロボットの推定位置とした.
% このようにして得られた値は,ループ検出の正誤判定に用いており,トポロジカルマップの構築には一切用いていない.
3D Lidar(Velodyne VLP-32MR),IMU(Xsense MTi-30),車輪オドメトリを用い,点群位置合わせに基づいてメトリックなロボット位置を求めた.
これらの情報は,トポロジカルマップの構築には一切用いていない.

\subsection{トポロジカルマップの構築}

% 本稿のトポロジカルマップの構築では,動画から画像列を作成する際の一定の時間間隔を1.0 [s],ノードの粗密を調整するパラメータ $node\_sim\_th$ = 0.80で統一した.

構築したトポロジカルマップと,ループ検出の正誤を評価する.
ループ検出の正誤判定には,\ref{sec:exp_conditions}節で述べたメトリックな情報を用いた.
本稿では,ループ検出の精度評価のために3つの手法でトポロジカルマップを構築,それぞれの結果を比較し,定性評価する.

\textbf{単純な閾値}:通常の全周画像を用いて特徴ベクトルを抽出し,クエリノードとターゲットノードのコサイン類似度が閾値を上回った場合にループ検出を行う.

\textbf{パノラマシフト + 単純な閾値}:パノラマシフトを行った全周画像を用いて特徴ベクトルを抽出し,単純な閾値と同様のループ検出を行う.

\textbf{パノラマシフト + 一貫性チェック}:パノラマシフトを行った全周画像を用いて特徴ベクトルを抽出し,空間的一貫性チェックを用いてループ検出を行う.

% \begin{itemize}
%   \item 単純な閾値: 通常の画像を用いて特徴ベクトルを抽出し,クエリノードとターゲットノードのコサイン類似度が $loop\_sim\_th$ を上回った場合にループ検出を行う.
%   \item パノラマシフト + 単純な閾値: パノラマシフトを行った画像を用いて特徴ベクトルを抽出し,クエリノードとターゲットノードのコサイン類似度が $loop\_sim\_th$ を上回った場合にループ検出を行う.
%   \item パノラマシフト + 一貫性チェック: パノラマシフトを行った画像を用いて特徴ベクトルを抽出し,空間的一貫性チェックを用いてループ検出を行う.
% \end{itemize}
% 全ての実験において,$loop\_sim\_th$ = 0.55に統一した.
% パノラマシフト + 一貫性チェックにおいては,$window\_size$ = 5,$count\_th$ = 2に統一し,$neighbor\_node\_num$ は走行経路により適応的に変更した.

% ループ検出の正誤判定には,\ref{sec:exp_conditions}節で述べたメトリックな情報を用いた.
% 各ノードに対して,ノード間のメトリックな距離が一定の閾値以内であれば同一地点と判定し,閾値以内のノード同士がループ検出された場合は True Positive (TP),閾値以上のノード同士がループ検出された場合は False Positive (FP)とした.
% また,閾値以内のノード同士がループ検出されなかった場合は False Negative (FN) とし,それら以外の場合は True Negative (TN) とした.
% ただし,単純な閾値では,時間的に近いノード同士がループ検出され,アークが追加されることがある.
% これはメトリックな距離は閾値の範囲内ではあるものの,トポロジカルマップの位相構造を崩すため,FPとして扱う.
% ループ検出の正誤判定に用いるメトリック距離の閾値は,3.0 [m] で統一した.

図\ref{fig:ex_ld_dkanjuuji3}〜\ref{fig:ex_ld_outdoor_crossroads}に,各走行経路において構築したトポロジカルマップを示す.
% 以降,屋内環境と屋外環境に分けて,トポロジカルマップの構築結果を比較する.
まず,屋内の走行経路である図\ref{fig:env_images}(a)とその実験結果である図\ref{fig:ex_ld_dkanjuuji3},\ref{fig:ex_ld_reverse_direction}を比較する.
% まず,屋内環境である図\ref{fig:ex_ld_dkanjuuji3},\ref{fig:ex_ld_reverse_direction}と図\ref{fig:env_images}(a)の走行経路を比較する.
% 両走行経路の単純な閾値,パノラマシフト + 単純な閾値では,実環境とは位相構造が異なるトポロジカルマップが構築されているのに対し,パノラマシフト + 一貫性チェックでは,実環境と同様の位相構造をもつトポロジカルマップが構築されていることが分かる.
% まず屋内交差において,パノラマシフト + 一貫性チェックでは,走行経路と同様な8の字型の位相構造が再現されている.
% 次に,屋内逆方向においても,パノラマシフト + 一貫性チェックでは,中央の通路が外側に出ているものの,走行経路と同様な位相構造が再現されている.
単純な閾値とパノラマシフト + 単純な閾値においては,TP なループ検出はあるものの,大きな FP のループが存在し,位相構造の崩れが顕著に見られる.
一方で,パノラマシフト + 一貫性チェックでは,他2手法で検出されたTPを残しつつ,FPを抑制している.

次に,屋外の走行経路である図\ref{fig:env_images}(b)とその実験結果である図\ref{fig:ex_ld_outdoor_crossroads}を比較する.
% 次に,屋外環境である図\ref{fig:ex_ld_outdoor_crossroads}と,図\ref{fig:env_images}(b)の走行経路を比較する.
単純な閾値では交差地点をループ検出できていないのに対し,パノラマシフト + 単純な閾値では交差地点をループ検出でき,大きなループを構築できている.
これはパノラマシフトによる再訪地点での視覚的類似度の向上により,ループ検出が可能になったと考えられる.
さらに,パノラマシフト + 一貫性チェックでは,パノラマシフトによるTPを維持しつつ,FPを抑制し,実環境と同様の位相構造をもつトポロジカルマップが構築されていることが分かる.
% 屋外逆方向においても,パノラマシフトによるTPの増加,一貫性チェックによるFPの抑制が見られる.
% また屋外交差と同様に,パノラマシフト + 一貫性チェックでは,TPを維持しつつFPを抑制し,位相的に妥当なトポロジカルマップが構築されていることが分かる.

これらの結果から提案手法は,パノラマシフトによる再訪地点の視覚的類似度の向上により TP なループ検出を増加させつつ,空間的一貫性チェックによりFPなループ検出を抑制し,実環境との位相構造の整合性をもつトポロジカルマップを構築できることが分かる.


\begin{figure}[t]
  \centering
  \begin{tabular}{c}
      \includegraphics[width=0.9\linewidth]{fig/env_dkan1f_2way.pdf}\\
      (a) 屋内交差(赤色)・屋内逆方向(青色)
  \end{tabular} 
  \begin{tabular}{c}
      \includegraphics[width=0.9\linewidth]{fig/env_outdoor_crossroads.pdf}\\
      (b) 屋外交差
  \end{tabular} 
  % \begin{tabular}{c}
  %     \includegraphics[width=0.7\linewidth]{fig/env_dkan_outdoor_inverse.pdf}\\
  %     (c) 屋外逆方向
  % \end{tabular} 
  \caption{実験環境と走行経路} \vspace*{-2mm}
  \label{fig:env_images}
\end{figure}


\begin{figure*}
  \centering
  \begin{tabular}{ccc}
    \includegraphics[width=0.25\linewidth]{fig/ex_ld_dkan1fjuuji3_simple.pdf} &
    \includegraphics[width=0.25\linewidth]{fig/ex_ld_dkan1fjuuji3_simple_with_shift.pdf} &
    \includegraphics[width=0.45\linewidth]{fig/ex_ld_dkan1fjuuji3_spatial.pdf} \\
    (a) 単純な閾値 & (b) パノラマシフト + 単純な閾値 & (c) パノラマシフト + 一貫性チェック(Ours)\\
  \end{tabular}
  \caption{屋内交差において構築したトポロジカルマップ}
  \label{fig:ex_ld_dkanjuuji3}

  \centering
  \begin{tabular}{ccc}
    \includegraphics[width=0.25\linewidth]{fig/ex_ld_reverse_direction_simple.pdf} &
    \includegraphics[width=0.25\linewidth]{fig/ex_ld_reverse_direction_simple_with_shift.pdf} &
    \includegraphics[width=0.45\linewidth]{fig/ex_ld_reverse_direction_spatial.pdf} \\
    (a) 単純な閾値 & (b) パノラマシフト + 単純な閾値 & (c) パノラマシフト + 一貫性チェック(Ours)\\
  \end{tabular}
  \caption{屋内逆方向において構築したトポロジカルマップ} 
  \label{fig:ex_ld_reverse_direction}

  \centering
  \begin{tabular}{ccc}
    \includegraphics[width=0.25\linewidth]{fig/ex_ld_outdoor_crossroads_simple.pdf} &
    \includegraphics[width=0.25\linewidth]{fig/ex_ld_outdoor_crossroads_simple_with_shift.pdf} &
    \includegraphics[width=0.45\linewidth]{fig/ex_ld_outdoor_crossroads_spatial.pdf} \\
    (a) 単純な閾値 & (b) パノラマシフト + 単純な閾値 & (c) パノラマシフト + 一貫性チェック(Ours)\\
  \end{tabular}
  \caption{屋外交差において構築したトポロジカルマップ}
  \label{fig:ex_ld_outdoor_crossroads}

%   \centering
%   \begin{tabular}{ccc}
%     \includegraphics[width=0.2\linewidth]{fig/ex_ld_outdoor_inverse_simple.pdf} &
%     \includegraphics[width=0.2\linewidth]{fig/ex_ld_outdoor_inverse_simple_with_shift.pdf} &
%     \includegraphics[width=0.35\linewidth]{fig/ex_ld_outdoor_inverse_spatial.pdf} \\
%     (a) 単純な閾値 & (b) パノラマシフト + 単純な閾値 & (c) パノラマシフト + 一貫性チェック (Ours) \\
%   \end{tabular}
%   \caption{屋外逆方向において構築したトポロジカルマップ}
%   \label{fig:ex_ld_outdoor_inverse}

\end{figure*}

\section{結言}

本稿では,全周画像の一定の基準のパノラマシフトにより,交差や逆方向での再訪においても視覚場所認識を可能とし,また空間的一貫性チェックによりループ検出の誤検出を抑制することで,実環境との位相構造の整合性を持つトポロジカルマップを構築する手法を提案した.

屋内外の3種類の走行経路で実験を行い定性評価した.
その結果,パノラマシフトにより再訪地点での視覚的類似度を向上させ,交差や逆方向での再訪においてもループ検出が可能となった.
さらに,空間的一貫性チェックにより誤検出を抑制し,実環境との位相構造の整合性を持つトポロジカルマップを構築できることを示した.
% 今後は,本稿で提案したトポロジカルマップを用いた自律走行の実現に取り組む.

\section*{謝辞}
本研究は,明治大学自律型ロボット研究クラスターの下で実施された.ここに御礼申し上げる

\begin{thebibliography}{00}
\bibitem{ref:robosym_ogiwara}
  荻原 大智, 高橋 尚起, 君塚 康介, 中尾 天哉, 原 祥尭, 黒田 洋司: 
  “視覚場所認識を用いたトポロジカルマップの構築”, 
  第 30回ロボティクスシンポジア予稿集, 2025.

\bibitem{ref:SII2026_nakao}
Takaya Nakao, Yoshitaka Hara, and Yoji Kuroda,
``Topological Mapping with Constrained Optimization based on Visual Place Recognition and Orientation Constraints'',
\textit{Proc. of IEEE/SICE Int. Sympo. on System Integration (SII)}, 2026.

\bibitem{ref:AnyLoc}
  Nikhil Keetha, Avneesh Mishra, Jay Karhade, Krishna Murthy Jatavallabhula, Sebastian Scherer, Madhava Krishna, and Sourav Garg: ``AnyLoc: Towards Universal Visual Place Recognition'',
  \textit{IEEE Robotics and Automation Letters}, vol. 9, no. 2, pp. 1286--1293, 2023.

% \bibitem{ref:Metric-Topological}
%   Sebastian Thrun:
%   ``Learning Metric-Topological Maps for Indoor Mobile Robot Navigation'',
%   \textit{Artificial Intelligence},vol. 99, no. 1, pp. 21--71, 1998.

% \bibitem{ref:The-Spatial-Semantic-Hierarchy}
%   Benjamin Kuipers:
%   ``The Spatial Semantic Hierarchy'',
%   \textit{Artificial Intelligence},vol. 119, no. 1-2, pp. 191--233, 2000.

% \bibitem{ref:ndt}
%   P. Biber, and W. Strasser: 
%   ``The normal distributions transform: a new approach to laser scan matching'',
%   \textit{Proc. of IEEE/RSJ Int. Conf. on Intelligent Robots and Systems (IROS)}, 2003.

% 参考文献書き方
% \bibitem{DrivingRecommendationMap}
%   Onozuka, Yuya and Matsumi, Ryosuke and Shino, Motoki:
%   ``Autonomous Mobile Robot Navigation Independent of Road Boundary Using Driving Recommendation Map'',
%   \textit{Proc. of IEEE/RSJ Int. Conf. on Intelligent Robots and Systems (IROS)}, 2021.
\end{thebibliography}
\vspace{12pt}
% \color{red}
% IEEE conference templates contain guidance text for composing and formatting conference papers. Please ensure that all template text is removed from your conference paper prior to submission to the conference. Failure to remove the template text from your paper may result in your paper not being published.

\end{document}
